\documentclass[hyperref={pdfpagelabels=false}]{beamer}
% la Hyperref opci\'on hyperref={pdfpagelabels=false} evita la advertencia :
% Package hyperref Warning: Option `pdfpagelabels' is turned off
% (hyperref)                because \thepage is undefined.
% Hyperref stopped earl)y
%
\usepackage[spanish]{babel}
\usepackage[utf8]{inputenc}
\usepackage{lmodern}
% elimina las siguientes advertencias:
% LaTeX Font Warning: Font shape `OT1/cmss/m/n' in size <4> not available
% (Font)              size <5> substituted on input line 22.
% LaTeX Font Warning: Size substitutions with differences
% (Font)              up to 1.0pt have occurred.
%

% cuando  \titel{$B!D(B} \author{$B!D(B} posicina desp\'ues \begin{document} ,
% aparece eso  advertencia: :
% Package hyperref Warning: Option `pdfauthor' has already been used,
% (hyperref) ...
% Por tanto posicina lo antes de  \begin{document}

\title{Prueba de oposición}
\author{Lucas Gabriel Vuotto}
\date{\today}


% Adicional intercala package{beamerthemeshadow}
\usepackage{beamerthemeshadow}
%  causa que elementos que aparece en el futuro
%  escribe ligero
%\beamersetuncovermixins{\opaqueness<1>{25}}{\opaqueness<2->{15}}
% funciona por tablas tamb\'\i en cuando aplica teTeX$B!D(B
\begin{document}


\begin{frame}
\titlepage %portada
\end{frame}


\begin{frame}
\frametitle{índice}
\tableofcontents
\end{frame}


\section{Entorno}

\begin{frame}
\frametitle{}
Acá iría en qué contexto se da el ejercicio. Posiblemente sería para cerrar
esta sección antes de empezar con circiutos con retroalimentación.
\end{frame}


\section{Objetivos}

\begin{frame}
\frametitle{Objetivos}
\begin{itemize}
  \item Repasar circuitos secuenciales.
  \item Demostrar que se puede hacer todo con \texttt{NAND} y mostrar las
  pros y contras de hacerlo.
  \item Implementar un full-adder.
  \item Transformarme en ayudante.
\end{itemize}
\end{frame}


\section{Ejercicio}

\begin{frame}
\frametitle{Enunciado}
\textbf{Ejercicio 11}
\begin{enumerate}
  \item Diseñar un \textit{full adder} de 1 bit usando sólo compuertas NAND.
  \item Suponiendo que todas las compuertas elementales tienen el mismo
  retardo \textit{(delay)} $t$, calcule el retardo total del circuito para
  producir todas sus señales de salida.
\end{enumerate}
\end{frame}


\begin{frame}
\frametitle{Diseñar un \textit{full adder} de 1 bit usando sólo compuertas
NAND}
Tabla de verdad de un \textit{full adder}
\begin{center}\begin{tabular}{| c | c | c || c | c |}
  \hline
  $e_0$ & $e_1$ & $c_e$ & $s$ & $c_s$ \\ \hline
    0   &   0   &   0   &  0  &   0   \\
    0   &   1   &   0   &  1  &   0   \\
    1   &   0   &   0   &  1  &   0   \\
    1   &   1   &   0   &  0  &   1   \\
    0   &   0   &   1   &  1  &   0   \\
    0   &   1   &   1   &  0  &   1   \\
    1   &   0   &   1   &  0  &   1   \\
    1   &   1   &   1   &  1  &   1   \\
  \hline
\end{tabular}\end{center}
\end{frame}


\begin{frame}
\frametitle{dummy}
Acá comentaría que tirarse a hacer el problema a lo cabeza puede ser un
quilombo, que lo mejor es partirlo en problemas chiquititos y empezar a
resolver desde ahí.
\end{frame}


\begin{frame}
\frametitle{Diseñar un \textit{full adder} de 1 bit usando sólo compuertas
NAND}
Para los que no recuerdan, esto es la tabla de verdad de un NAND:
\begin{center}\begin{tabular}{| c | c || c |}
  \hline
  $e_0$ & $e_1$ & $e_0$ \texttt{NAND} $e_1$ \\ \hline
    0   &   0   &             1             \\
    0   &   1   &             1             \\
    1   &   0   &             1             \\
    1   &   1   &             0             \\
  \hline
\end{tabular}\end{center}
\end{frame}


\begin{frame}
\frametitle{Diseñar un \textit{full adder} de 1 bit usando sólo compuertas
NAND}
\textit{half adder}
\\ ¡Grafiquito!
\end{frame}


\begin{frame}
\frametitle{Diseñar un \textit{full adder} de 1 bit usando sólo compuertas
NAND}
\textit{full adder}
\\ ¡Grafiquito! Half adder + Half adder, salida del primero con carry in, or
de los carries de salida.
\end{frame}


\begin{frame}
\frametitle{Diseñar un \textit{full adder} de 1 bit usando sólo compuertas
NAND}
\textit{full adder}
\\ ¡Grafiquito full adder en la derecha! tabla de verdad de un AND en la
izquierda $\Rightarrow$ nos damos cuenta que es un NAND es un NOT-AND.
\end{frame}


\begin{frame}
\frametitle{Diseñar un \textit{full adder} de 1 bit usando sólo compuertas
NAND}
\textit{full adder}
\\ ¡Grafiquito full adder en la derecha! hacemos una tablita para hacer un
NOT con un NAND.
\end{frame}


\begin{frame}
\frametitle{Diseñar un \textit{full adder} de 1 bit usando sólo compuertas
NAND}
\textit{full adder}
\\ ¡Grafiquito full adder en la derecha! updateamos la tablita del AND con
la nueva info.
\end{frame}


\begin{frame}
\frametitle{Diseñar un \textit{full adder} de 1 bit usando sólo compuertas
NAND}
\textit{full adder}
\\ ¡Grafiquito full adder en la derecha! ahora hacemos un OR con NANDs,
pensandolo un poquito \textcolor{red}{$\Rightarrow$ \textbf{ARMAR LA IDEA}}.
\end{frame}


\begin{frame}
\frametitle{Diseñar un \textit{full adder} de 1 bit usando sólo compuertas
NAND}
\textit{full adder}
\\ ¡Grafiquito full adder en la derecha! cerramos haciendo un XOR con NANDs,
también con alguna idea de cómo inferirlo.
\end{frame}


\begin{frame}
\frametitle{Diseñar un \textit{full adder} de 1 bit usando sólo compuertas
NAND}
\textit{full adder}
\\ ¡Grafiquito full adder en la derecha! ahora vamos reemplazando en el
grafico del full adder las compuertas AND, OR y XOR por sus equivalentes en
NANDs.
\end{frame}


% \begin{frame}
% \frametitle{Diseñar un \textit{full adder} de 1 bit usando sólo compuertas
% NAND}
% \textit{full adder}
% \\ ¡Grafiquito full adder en la derecha!
% \end{frame}
% 
% 
% \begin{frame}
% \frametitle{Diseñar un \textit{full adder} de 1 bit usando sólo compuertas
% NAND}
% \textit{full adder}
% \\ ¡Grafiquito full adder en la derecha!
% \end{frame}
% 
% 
% \begin{frame}
% \frametitle{Diseñar un \textit{full adder} de 1 bit usando sólo compuertas
% NAND}
% \textit{full adder}
% \\ ¡Grafiquito full adder en la derecha!
% \end{frame}
% 
% 
% \begin{frame}
% \frametitle{Diseñar un \textit{full adder} de 1 bit usando sólo compuertas
% NAND}
% \textit{full adder}
% \\ ¡Grafiquito full adder en la derecha!
% \end{frame}
% 
% 
% \begin{frame}
% \frametitle{Diseñar un \textit{full adder} de 1 bit usando sólo compuertas
% NAND}
% \textit{full adder}
% \\ ¡Grafiquito full adder en la derecha!
% \end{frame}
% 
% 
% \begin{frame}
% \frametitle{Diseñar un \textit{full adder} de 1 bit usando sólo compuertas
% NAND}
% \textit{full adder}
% \\ ¡Grafiquito full adder en la derecha!
% \end{frame}
% 
% 
% \begin{frame}
% \frametitle{Diseñar un \textit{full adder} de 1 bit usando sólo compuertas
% NAND}
% \textit{full adder}
% \\ ¡Grafiquito full adder en la derecha!
% \end{frame}
% 
% 
% \begin{frame}
% \frametitle{Diseñar un \textit{full adder} de 1 bit usando sólo compuertas
% NAND}
% \textit{full adder}
% \\ ¡Grafiquito full adder en la derecha!
% \end{frame}
% 
% 
% \begin{frame}
% \frametitle{Diseñar un \textit{full adder} de 1 bit usando sólo compuertas
% NAND}
% \textit{full adder}
% \\ ¡Grafiquito full adder en la derecha!
% \end{frame}


\begin{frame}
\frametitle{Sabiendo que las compuertas elementales tienen un delay de $t$,
calcular el retardo total del circuito}
\textit{Comparativa con un gráfico de cada lado}
\\ Contar numeritos, hablar sobre pese a que te resuelve la vida
especializarte en un solo tipo de compuerta, tenés que meter más cosas en el
medio y pueden ser menos eficientes los circuitos
\textcolor{red}{$\Rightarrow$ \textit{¿es chamuyo eso?}}
\end{frame}


\begin{frame}
¿Preguntas?
\end{frame}


\end{document}
