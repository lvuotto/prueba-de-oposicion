\documentclass[mathserif,hyperref={pdfpagelabels=false}]{beamer}

\usepackage[spanish]{babel}
\usepackage[utf8]{inputenc}
\usepackage{amsmath}
\usepackage{array}
\usepackage{adjustbox}
\usepackage{lmodern}

% elimina las siguientes advertencias:
% LaTeX Font Warning: Font shape `OT1/cmss/m/n' in size <4> not available
% (Font)              size <5> substituted on input line 22.
% LaTeX Font Warning: Size substitutions with differences
% (Font)              up to 1.0pt have occurred.

\title{Prueba de oposición}
\author{Lucas Gabriel Vuotto}
\date{\today}


% Adicional intercala package{beamerthemeshadow}
\usepackage{beamerthemeshadow}
%  causa que elementos que aparece en el futuro
%  escribe ligero
%\beamersetuncovermixins{\opaqueness<1>{25}}{\opaqueness<2->{15}}
% funciona por tablas tamb\'\i en cuando aplica teTeX$B!D(B
\begin{document}


\begin{frame}
\titlepage %portada
\end{frame}


\begin{frame}
\frametitle{índice}
\tableofcontents
\end{frame}


\section{Entorno}

\begin{frame}
\frametitle{Entorno}
Para este ejercicio, se asume que los alumnos conocen las compuertas lógicas
básicas y cómo diseñar circuitos combinatorios.

Se recomienda dar este ejercicio a modo integratorio antes de comenzar con
circuitos secuenciales.
\end{frame}


\section{Objetivos}

\begin{frame}
\frametitle{Objetivos}
\begin{itemize}
  \item Repasar circuitos secuenciales.
  \item Mostrar que se puede hacer todo con \texttt{NAND}s.
  \item Hacer el mejor esfuerzo para ser ayudante.
\end{itemize}
\end{frame}


\section{Ejercicio}

\begin{frame}
\frametitle{Enunciado}
\textbf{Ejercicio 11}
\begin{enumerate}
  \item Diseñar un \textit{full adder} de 1 bit usando sólo compuertas NAND.
  \item Suponiendo que todas las compuertas elementales tienen el mismo
  retardo \textit{(delay)} $t$, calcule el retardo total del circuito para
  producir todas sus señales de salida.
\end{enumerate}
\end{frame}


\begin{frame}
\frametitle{\small Diseñar un \textit{full adder} de 1 bit usando sólo
compuertas NAND}
Tabla de verdad de un \textit{full adder}
\begin{center}\begin{tabular}{| c | c | c || c | c |}
  \hline
  $e_0$ & $e_1$ & $c_e$ & $s$ & $c_s$ \\ \hline
    0   &   0   &   0   &  0  &   0   \\
    0   &   1   &   0   &  1  &   0   \\
    1   &   0   &   0   &  1  &   0   \\
    1   &   1   &   0   &  0  &   1   \\
    0   &   0   &   1   &  1  &   0   \\
    0   &   1   &   1   &  0  &   1   \\
    1   &   0   &   1   &  0  &   1   \\
    1   &   1   &   1   &  1  &   1   \\
  \hline
\end{tabular}\end{center}
\end{frame}


\begin{frame}
\frametitle{dummy}
Acá comentaría que tirarse a hacer el problema a lo cabeza puede ser un
quilombo, que lo mejor es partirlo en problemas chiquititos y empezar a
resolver desde ahí.
\end{frame}


\begin{frame}
\frametitle{\small Diseñar un \textit{full adder} de 1 bit usando sólo
compuertas NAND}
Para los que no recuerdan, esto es la tabla de verdad de un NAND:
\begin{center}\begin{tabular}{| c | c || c |}
  \hline
  $e_0$ & $e_1$ & $e_0$ \texttt{NAND} $e_1$ \\ \hline
    0   &   0   &             1             \\
    0   &   1   &             1             \\
    1   &   0   &             1             \\
    1   &   1   &             0             \\
  \hline
\end{tabular}\end{center}
\end{frame}


\begin{frame}
\frametitle{\small Diseñar un \textit{full adder} de 1 bit usando sólo
compuertas NAND}
\textit{Half adder}
\begin{figure}[htp]
  \includegraphics[scale=0.9]{halfadder.pdf}
\end{figure}
\end{frame}


\begin{frame}
\frametitle{\small Diseñar un \textit{full adder} de 1 bit usando sólo
compuertas NAND}
\textit{Full adder}
\begin{figure}[htp]
  \includegraphics[scale=0.9]{fulladder.pdf}
\end{figure}
\end{frame}


\begin{frame}
\frametitle{\small Diseñar un \textit{full adder} de 1 bit usando sólo
compuertas NAND}
\textit{Full adder}
\begin{columns}[T]
  \column{.5\textwidth}
  \begin{figure}[htp]
    \includegraphics[scale=0.9]{fulladder.pdf}
  \end{figure}
  \column{.5\textwidth}
  \begin{center}\begin{tabular}{| c | c || c |}
    \hline
    $e_0$ & $e_1$ & $e_0$ \texttt{AND} $e_1$ \\ \hline
      0   &   0   &             0            \\
      0   &   1   &             0            \\
      1   &   0   &             0            \\
      1   &   1   &             1            \\
    \hline
  \end{tabular}\end{center}
  \pause
  \begin{center}\begin{tabular}{| c | c || c |}
    \hline
    $e_0$ & $e_1$ & $e_0$ \texttt{NAND} $e_1$ \\ \hline
      0   &   0   &              1            \\
      0   &   1   &              1            \\
      1   &   0   &              1            \\
      1   &   1   &              0            \\
    \hline
  \end{tabular}\end{center}
\end{columns}
\end{frame}


\begin{frame}
\frametitle{\small Diseñar un \textit{full adder} de 1 bit usando sólo
compuertas NAND}
\textit{Full adder}
\begin{columns}[T]
  \column{.5\textwidth}
  \begin{figure}[htp]
    \includegraphics[scale=0.9]{fulladder.pdf}
  \end{figure}
  \column{.5\textwidth}
  \begin{center}\begin{tabular}{| c | c || c |}
    \hline
    $e$ & $1$ & NOT $e$ \\ \hline
     0  &  1  &    1    \\
     1  &  1  &    0    \\
    \hline
  \end{tabular}\end{center}
\end{columns}
\end{frame}


\begin{frame}
\frametitle{\small Diseñar un \textit{full adder} de 1 bit usando sólo
compuertas NAND}
\textit{full adder}
\begin{columns}[T]
  \column{.5\textwidth}
  \begin{figure}[htp]
    \includegraphics[scale=0.9]{fulladder.pdf}
  \end{figure}

  \column{.5\textwidth}
  %\adjustbox{max width=.5\textwidth}{
  \small \begin{center}\begin{tabular}{| c | c || c | c || c |}
    \hline
    $e_0$ & $e_1$ & $e_0$ NAND $e_1$ & 1 &  $e_0$ NAND $e_1$ NAND $1$ \\
    \hline
      0   &   0   &         1        & 1 &               0            \\
      0   &   1   &         1        & 1 &               0            \\
      1   &   0   &         1        & 1 &               0            \\
      1   &   1   &         0        & 1 &               1            \\
    \hline
  \end{tabular}\end{center}
  %}
\end{columns}
% \\ ¡Grafiquito full adder en la derecha! updateamos la tablita del AND con
% la nueva info.
\end{frame}


\begin{frame}
\frametitle{\small Diseñar un \textit{full adder} de 1 bit usando sólo
compuertas NAND}
\textit{full adder}
\begin{columns}[T]
  \column{.5\textwidth}
  \begin{figure}[htp]
    \includegraphics[scale=0.9]{fulladder.pdf}
  \end{figure}

  \column{.5\textwidth}
  \begin{center}\begin{tabular}{| c | c || c |}
    \hline
    $e_0$ & $e_1$ & $e_0$ OR $e_1$ \\ \hline
      0   &   0   &        0       \\
      0   &   1   &        1       \\
      1   &   0   &        1       \\
      1   &   1   &        1       \\
    \hline
  \end{tabular}\end{center}
  \pause
  \begin{center}\begin{tabular}{| c | c || c |}
    \hline
    $e_0$ & $e_1$ & $e_0$ NAND $e_1$ \\ \hline
      1   &   1   &         0        \\
      1   &   0   &         1        \\
      0   &   1   &         1        \\
      0   &   0   &         1        \\
    \hline
  \end{tabular}\end{center}
\end{columns}
% \\ ¡Grafiquito full adder en la derecha! ahora hacemos un OR con NANDs,
% pensandolo un poquito \textcolor{red}{$\Rightarrow$ \textbf{ARMAR LA IDEA}}.
\end{frame}


\begin{frame}
\frametitle{\small Diseñar un \textit{full adder} de 1 bit usando sólo
compuertas NAND}
\textit{full adder}
\begin{columns}[T]
  \column{.5\textwidth}
  \begin{figure}[htp]
    \includegraphics[scale=0.9]{fulladder.pdf}
  \end{figure}

  \column{.5\textwidth}
  Recordemos, $e_0$ XOR $e_1$ = ($e_0$ OR $e_1$) AND ($\lnot e_0$ OR $\lnot
  e_1$)
\end{columns}
% \\ ¡Grafiquito full adder en la derecha! cerramos haciendo un XOR con NANDs,
% también con alguna idea de cómo inferirlo.
\end{frame}


\begin{frame}
\frametitle{\small Diseñar un \textit{full adder} de 1 bit usando sólo
compuertas NAND}
\textit{full adder}
\begin{columns}[T]
  \column{.5\textwidth}
  \begin{figure}[htp]
    \includegraphics[scale=0.9]{fulladder.pdf}
  \end{figure}

  \column{.5\textwidth}
  \begin{figure}[htp]
    \includegraphics<1>[scale=0.9]{halfadder.pdf}
    \includegraphics<2>[scale=0.9]{fulladder.pdf}
  \end{figure}
\end{columns}
% \\ ¡Grafiquito full adder en la derecha! ahora vamos reemplazando en el
% grafico del full adder las compuertas AND, OR y XOR por sus equivalentes en
% NANDs.
\end{frame}


% \begin{frame}
% \frametitle{Diseñar un \textit{full adder} de 1 bit usando sólo compuertas
% NAND}
% \textit{full adder}
% \\ ¡Grafiquito full adder en la derecha!
% \end{frame}
% 
% 
% \begin{frame}
% \frametitle{Diseñar un \textit{full adder} de 1 bit usando sólo compuertas
% NAND}
% \textit{full adder}
% \\ ¡Grafiquito full adder en la derecha!
% \end{frame}
% 
% 
% \begin{frame}
% \frametitle{Diseñar un \textit{full adder} de 1 bit usando sólo compuertas
% NAND}
% \textit{full adder}
% \\ ¡Grafiquito full adder en la derecha!
% \end{frame}
% 
% 
% \begin{frame}
% \frametitle{Diseñar un \textit{full adder} de 1 bit usando sólo compuertas
% NAND}
% \textit{full adder}
% \\ ¡Grafiquito full adder en la derecha!
% \end{frame}
% 
% 
% \begin{frame}
% \frametitle{Diseñar un \textit{full adder} de 1 bit usando sólo compuertas
% NAND}
% \textit{full adder}
% \\ ¡Grafiquito full adder en la derecha!
% \end{frame}
% 
% 
% \begin{frame}
% \frametitle{Diseñar un \textit{full adder} de 1 bit usando sólo compuertas
% NAND}
% \textit{full adder}
% \\ ¡Grafiquito full adder en la derecha!
% \end{frame}
% 
% 
% \begin{frame}
% \frametitle{Diseñar un \textit{full adder} de 1 bit usando sólo compuertas
% NAND}
% \textit{full adder}
% \\ ¡Grafiquito full adder en la derecha!
% \end{frame}
% 
% 
% \begin{frame}
% \frametitle{Diseñar un \textit{full adder} de 1 bit usando sólo compuertas
% NAND}
% \textit{full adder}
% \\ ¡Grafiquito full adder en la derecha!
% \end{frame}
% 
% 
% \begin{frame}
% \frametitle{Diseñar un \textit{full adder} de 1 bit usando sólo compuertas
% NAND}
% \textit{full adder}
% \\ ¡Grafiquito full adder en la derecha!
% \end{frame}


\begin{frame}
\frametitle{\small Sabiendo que las compuertas elementales tienen un delay
de $t$, calcular el retardo total del circuito}
\textit{Comparativa con un gráfico de cada lado}
\\ Contar numeritos, hablar sobre pese a que te resuelve la vida
especializarte en un solo tipo de compuerta, tenés que meter más cosas en el
medio y pueden ser menos eficientes los circuitos
\textcolor{red}{$\Rightarrow$ \textit{¿es chamuyo eso?}}
\end{frame}


\begin{frame}
¿Preguntas?
\end{frame}


\end{document}
